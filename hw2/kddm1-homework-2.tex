\documentclass[a4paper,10pt]{article}\setlength{\textheight}{10in}\setlength{\textwidth}{6.5in}\setlength{\topmargin}{-0.125in}\setlength{\oddsidemargin}{-.2in}\setlength{\evensidemargin}{-.2in}\setlength{\headsep}{0.2in}\setlength{\footskip}{0pt}\usepackage{amsmath}\usepackage{fancyhdr}\usepackage{enumitem}\usepackage{hyperref}\usepackage{xcolor}\usepackage{graphicx}\pagestyle{fancy}

\lhead{Name: \rule{5cm}{0.5pt}}
\chead{M.Number: \rule{2cm}{0.5pt}}
\rhead{KDDM1 VO (INP.31101UF)}
\fancyfoot{}

\begin{document}
\begin{enumerate}[topsep=0mm, partopsep=0mm, leftmargin=*]

{\color{blue}
\item\textit{Distance Functions}. Given the dataset ``distance-function-dataset.csv'' (available in TeachCenter), select or develop a suitable distance function to compare instances (row), based on the values of the features (columns).
\begin{enumerate}
	\item On what observations from the dataset do you base your decisions? (bullet points)
	\item Would you conduct any additional feature engineering steps?
	\item What distance function do you choose? (in case of a custom one, please provide a description/pseudocode/...)
	\item Would the distance function depend on the succeeding processing, e.g., different function for PCA, DBSCAN, or SVM?
\end{enumerate}
}

%%% Your answer here


{\color{blue}
\newpage\item\textit{Dimensionality Reduction}. Consider a dataset of 100 dimensions/features (real numbers), and the goal is to derive a 2D visualisation of the dataset. 
\begin{enumerate}
	\item What would be a suitable approach if the dependencies in the data are all linear?
	\item What would be a suitable approach if there are density-based local structures in the data?
	\item What would be a suitable approach if most of the features are Gaussain noise?
	\item What types of noise are there and how do they affect the dimensionality reduction?
\end{enumerate}
}

%%% Your answer here



{\color{blue}
\newpage\item \textit{Clustering}. Given the dataset ``clustering-dataset.csv'' (available in TeachCenter), which consists of observations of 5 dimensions, the goal is to find the groups of rows that form clusters.
\begin{enumerate}
	\item Which methods did you apply to find the clusters, and why? (bullet points)
	\begin{itemize}
		\item Describe pre-processing steps (if conducted)
		\item Describe what distance measures you have chosen
		\item Describe how you determine the number of clusters
	\end{itemize}
	\item What clusters did you find and how would you describe the distribution of the points within each cluster?
	\begin{itemize}
		\item Describe each found cluster, including shape, and amount of points within the cluster.
	\end{itemize}
\end{enumerate}
}

%%% Your answer here



{\color{blue}
\newpage\item\textit{Classification}. Select 3 classification algorithms of your choice that should be diverse (i.e., not based on the same underlying principles).
\begin{enumerate}
	\item For each algorithm list the main assumptions (e.g., on the data characteristics, types of dependencies). (bullet points) 
	\item For each algorithm list 1-2 application scenarios, where these assumptions are being met.
\end{enumerate}
}

%%% Your answer here


\end{enumerate}
\end{document}
